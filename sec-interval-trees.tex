\section{Interval trees for orthogonal stabbing queries}
\label{sec:interval-trees}

In certain geometric applications like planar graphs, we might be interested in reporting the set of line segments $L$ associated with a window query  $w = [x_0, x_1] \times [y_0, y_1]$. 
%
This set $L$ contains all line segments with at least one end point inside $w$, or those segments that pass through the window $w$. 

We can report all line segments with at least one endpoint inside $w$ using algorithms and techniques from Section \ref{sec:range-queries}. 
%
We can use 2 instances of 3-sided query i.e. $ q_0 = [x_0, x_1] \times [y_0, +\infty ]  $ and $q_1 = [x_0, x_1] \times [-\infty,y_1]$, and then report the intersections of the two queries in time $O(\log n + k_0) + O(\log n + k_1) $. 
%
However, in the worst case scenario, the two queries may report all $n = k_0 + k_1$ endpoints while the intersection may contain only one element.
%
By constructing 2D range trees with storage $O(n\log n)$, and time $O(\log^2 n + k)$ we can report all line segments with ends points in $w$. Applying fractional cascading can reduce the query time further to $O(\log n +k)$ which is faster than $(\log n + n)$.

What is left is how to handle efficiently those line segments that pass through our window query $w$? In other words, we are looking for line segments that stab two boundaries of our window query $w$. If we know how to report all line segments that stab one boundary,  say $[x_0, x_1]$, of $w$, we can easily check if they stab any of the remaining boundaries. In this section, we present a data structure that can handle stabbing queries for orthogonal line segments in $O(\log n + k)$ time, and in the next section, we show how report stabbing queries for slanted line segments in $O(\log n + k)$. 

ADD PICTURE HERE OF WINDOW QUERY
